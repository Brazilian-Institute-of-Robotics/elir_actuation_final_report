\begin{thesisresumo}
Este documento contempla a descrição das etapas do desenvolvimento do projeto de Theoprax de Conclusão de curso, ELIR, robô autônomo de inspeção de linhas de transmissão, atendendo aos objetivos gerais e específicos e aos requisitos estabelecidos pelo cliente, tendo em vista a necessidade do projeto num cenário tanto comercial quanto acadêmico. Durante o desenvolvimento do projeto foi necessário realizar o estudo de conceitos de robótica, bem como os softwares necessários para implementação das funcionalidades, também estudadas e definidas pelo grupo. Em paralelo ao desenvolvimento das funcionalidades diversos testes foram realizados para validação dos conceitos e verificação de erros, em etapas de testes individuais partindo para a etapa de testes integrados. Os conceitos estudados e desenvolvidos pelo grupo durante todo o projeto fazem parte de uma grande contribuição tecnológica para a área de robótica e engenharia, sendo um projeto enriquecedor tanto para a equipe envolvida no desenvolvimento quanto para as gerações futuras interessadas no desenvolvimento tecnológico em robótica.   




% use de três a cinco palavras-chave

\textbf{Palavras-chave}: Robô de Inspeção, Linhas de Transmissão, Navegação, Cinemática Inversa, Manipuladores

\end{thesisresumo}
