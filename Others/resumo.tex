\begin{thesisresumo}
	
Devido às dimensões continentais do Brasil e a distância entre os pontos de geração de energia e os grandes centros consumidores, se fazem necessárias muitas linhas de transmissão em todo o território brasileiro. Muitas dessas linhas estão operando a muitos anos, com isso, a necessidade de manutenções se torna mais frequente. Atualmente essas manutenções são realizadas utilizando aeronaves tripuladas e tem como características: altos custos para companhias de energia e alta periculosidade para os operadores de manutenção. Uma alternativa à essas operações é utilizar um robô autônomo.
O presente trabalho traz o desenvolvimento da movimentação do robô \textit{ELIR} (\textit{Electrical Line Inspection Robot}), implementando as seguintes funcionalidades: Atuação; Planejamento do movimento; Gerenciamento de Energia e Checagem de Integridade do Sistema. Para essa tarefa foi utilizado o \textit{framework ROS} (\textit{Robot Operating System}) e desenvolvidas rotinas de ultrapassagem utilizando simulações nos softwares \textit{MoveIt!} e \textit{Gazebo}. Divididos em unitários e integrados, testes foram realizados para validar o desenvolvimento. Os unitários mostraram como os componentes funcionavam e os integrados, o comportamento do robô como um sistema completo.   




% use de três a cinco palavras-chave

\textbf{Palavras-chave}: Robô de Inspeção, Linhas de Transmissão, Navegação, Cinemática Inversa, Manipuladores

\end{thesisresumo}
