\begin{thesisabastract}
Due to the continental dimensions of Brazil and the distance between the energy generation points and the large consumer centers, many transmission lines are needed throughout the Brazilian territory. Many of these lines are operating for many years, so the need for maintenance becomes more frequent. Currently, these maintenances are performed using manned aircraft and have the following characteristics: high costs for energy companies and high risk for maintenance operators. An alternative to these operations is to use a standalone robot.
The present work brings the development of the robot movement ELIR (Electrical Line Inspection Robot), implementing the following functionalities: Actuation; Motion planning; Power Management and System Integrity Check. For this task was used the framework ROS (Robot Operating System) and developed overtaking routines using simulations in the software MoveIt! and Gazebo. Divided into unitary and integrated, tests were performed to validate the development. The unitaries showed how the components worked and the integrated ones, the behavior of the robot as a complete system.


\ \\

\textbf{Keywords}: Inspection Robot,Transmission Lines,Navigation,Inverse Kinematics,Manipulators

\end{thesisabastract}
