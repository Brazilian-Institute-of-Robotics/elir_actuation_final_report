\chapter{Concluding Remarks}\label{chpconclusion}

The world has known about AIDS for more than twenty years. During this time the disease
has spread to every corner of the world and in the worst affected countries it has set
back human progress by decades. Turning back the HIV/AIDS epidemic is a task beyond
individual effort, no matter how outstanding or heroic. It requires governments, nations,
regions and communities to come together in concerted and co-ordinated action.

Consistent and courageous policies can halt the spread of the disease and let those
infected with HIV live a normal and dignified life by using four essential elements:
prevention, treatment, human rights and resources. Such policies have proved to be very
effective in controlling the transmission rate as well as extending the life of those
infected. For example, Brazil has more than halved its predicted AIDS epidemic by
ensuring free treatment. The support given by governments is fundamental to making the
population feel encouraged to accept voluntary and confidential testing (VCT), thus
improving the notification of HIV infection in earlier stages that otherwise would be
hidden and passed on to other people.

In the first years of the HIV epidemic, there was a great urge to look for historical
models as a means of dealing with the AIDS pandemic by finding its similarities with
other well known STDs. However HIV requires sophisticated approaches, capable not only of
recognising how HIV is like past epidemics, but precisely identifying the ways in which
it is different. Scientists have been working around the clock for more than two decades
to develop an effective cure for HIV, but despite the success of expensive HAART
interventions in slowing down the progress of the disease, no cure is in sight. The best
hope for controlling the HIV/AIDS pandemic lies in the development of an effective HIV
vaccine.

Traditional epidemiological models largely disregard the complex patterns and structures
of intimate contacts, yet they are the standard for quantifying the spread of HIV
worldwide. These models deal with well-mixed populations, where sub-groups interact in
proportion to their sizes ignoring the population structure altogether, or treat the
population as spatially distributed in a medium as elements in a lattice. However, real
populations rarely fall into either of these categories, being neither well mixed nor
lattice. In a sexual network, individuals are strongly heterogeneous in their sexual
preferences and the length of partnerships is extremely variable as are the frequency of
their sexual activities.

Social networks analysis embodies the social structure by addressing issues of
centrality, which individuals are best connected to others, who is more influential and
connectivity. The identity and precise nature of the individuals is down played, or even
suppressed, in the hope of uncovering deeper laws. Furthermore, traditional network
analysts still have problems with dynamics and networks are treated as the frozen
embodiment of social forces. All one needs to do is collect network data, measure the
right properties and all will be revealed. Unfortunately as life experience tells,
personal and social lives are always under construction, continuously changing
dynamically. This static structure analysis can be thought of as snapshots taken during
this ongoing process of evolution.

In a dynamic view of networks, existing structures can only be properly understood in
terms of the nature of the processes that led to them. In response to small
perturbations, both the network structure and the pattern of activity on the network will
change. Furthermore, each kind of decision helps set the context in which subsequent
decisions must be made. The HIVacSim model defines a network of spontaneous order, where
neighbours are defined by geography and social rules. Its connectivity gives rise to rich
forms of self-organisation, people live in a small world and diseases or information are
always dynamically transmitted over the network.

\section{Small World Networks}

The original small world models \cite{Watts1998,Newman1999} have proved to be unsuitable
for modelling social networks and the spread of infectious diseases for three main
reasons. Firstly, the network vertices are equally considered for rewiring or shortcuts,
which is unlike the real world as the level of interaction that an individual possesses
will not be the same as other individuals. Secondly, they do not characterise the
ties/link properties of edges, essential to capture the social behaviour of an individual
within the network. And finally, connections are equally weighted or the network is
strongly connected.

Sexual networks are rarely fully connected (\ref{swnproperties}). Because of it the
original formulation characterising the small world network global connectivity due to
Watts and Strogatz \cite{Watts1998} cannot be used as it requires a fully connected
network. Therefore, a small world network is better characterised by the network
efficiency introduced by Latora and Marchiori \cite{Latora2001} for measuring the global
and local efficiency of a network (\ref{netefficiency}). From this perspective a small
world network is a network with high global and high local efficiency, propagating
diseases or information very efficiently both on a global and on a local scale.

In order to accommodate differential selectivity and sexual behaviour, network properties
such as performance and concurrency, which are fundamental for the modelling of human
sexual network and the spread of HIV must be included. We have modified the original
small world model so that the connectedness restriction is relaxed, conditions that two
vertices must hold in order to be connected are added, and properties for each vertex and
edge are added in order to capture sexual behaviour changes.

HIVacSim model represents a small world network and is consistent with the general
definition of the small world theory (\ref{netefficiency}, \ref{swnclustering}). The
short paths represent the \emph{global efficiency} and provide high-speed communication
channels between distant parts of the network. This characteristic facilitates any
dynamical processes that require global coordination, transmission of information or
propagation of infectious disease. The \emph{local efficiency} provides short distance
communication, enabling high-speed propagation of information or disease through local
clusters within the population.

The degree distribution quantifies the size of the lists of acquaintances, hence the
popularity of individuals within a population. The small world randomness parameter
\emph{p} has a remarkable effect on the degree distribution of individuals. As the
randomness value \emph{p} increases, the location, shape and scale of the degree
distribution also change following a non-linear scale.

The effects of topology on network efficiency must not be overlooked. Traditional social
networks and epidemiological models ignore the dynamics of information and infectious
diseases spread through geography; initially infected individuals or information holders
are uniformly distributed within the population. In such a case, one is likely to get
infected or acquire the knowledge from anywhere in the network with the same probability,
independent of topology or one's geographical location. However, life experience suggests
that knowledge is not easy to find and information holders are not uniformly distributed.
The spread of infectious diseases on the other hand can vary enormously by geography
(e.g. HIV prevalence).

In a dynamic geographically clustered distribution of initially infected individuals or
sources of information, topology matters and has a distinct effect on network efficiency.
The probability of meeting an infected individual close to someone will depend upon one's
social and geographical location within the network. Information or infectious diseases
are dynamically transmitted geographically through social interactions.


The original \emph{circle topology} of the small world network has the lowest network
efficiency. \emph{Topology free} networks typical of traditional social networks and
epidemiological models with no geographical considerations, have the highest network
efficiency, however they completely ignore the geographical distribution of individuals
and consequently are not a realistic representation of the real world. \emph{Spherical
topology} is a intuitive representation of the real world and its network efficiency lies
between regularity (circle) and randomness (free). The initial distribution of infection
or sources of information also has a significant effect on network efficiency within the
same topology.

HIVacSim allows both geographical and social distances to be taken into account as part
of the dynamics of social interactions. Actors are uniquely represented by their personal
and social characteristics within the system. They can choose their partners and have
different sexual behaviour when in a partnership. The making and breaking of network ties
take place over time, vertices can be added or removed at any point in time and the
strength of the interactions are governed by social, cultural and structural rules. Thus,
there is dynamics on the network and the dynamics of social networks and spread of
diseases or information can be better understood.


\section{Epidemiology in a Small World}

The small world network theory may explain in part why HIV has managed to spread itself
to every corner of the world, infecting and killing people from all ethnic and social
backgrounds, surviving like no other disease has ever done in the same proportion and
time scale. An infectious disease or information needs only a small amount of randomness
$(p \sim 0.2)$ on network interactions in order for it to spread efficiently on a local
and global scale.

The randomness parameter of the small world (probability of casual partnership) has a
direct and non-linear effect on HIV transmission (\ref{sweffecthiv}). The network
efficiency peaks at about 90\% of randomness and not at 100\% as one might expect. This
effect is caused by the differences in sexual behaviour regarding safe sex practices
between stable and casual partnerships. The expected small world effect on the growth of
the HIV epidemic can be achieved by assuming a homogenous sexual behaviour towards safe
sex practices for both stable and casual partnerships.

Compartmental models such as UNAIDS' EPP, used to estimate the HIV epidemic worldwide,
can be easily represented and evaluated within HIVacSim (\ref{hivacsimsir}). However
unlike this traditional family of epidemiological models, HIVacSim identifies the
different routes of infection (Table \ref{outputdata}), which is fundamental for
understanding the spread of an epidemic and planning better intervention strategies. It
also quantifies the characteristics of the network (Table \ref{netproperties}) through
which the transmission occurs, providing a detailed local and global view of the
epidemic's development. This enables targeted intervention strategies to be planned
globally and delivered at different levels within the population such as a local
community or specific risk group.

The course of the HIV/AIDS epidemic is changing rapidly at the level of the virus,
treatment availability and within populations at risk. Nevertheless, sexual transmission
of HIV remains the main force behind the epidemic. HIVacSim measures the impact of
selective mixing, variations in partnership characteristics such as length and
overlapping, and sexual behaviour changes towards safer sex practices, the single most
effective method of preventing HIV infection.

The cultural and social traditions of a community play an important role in the spread of
HIV. In modern societies adultery may not be a norm but is usually tolerated. The results
presented in Section \ref{swnconcurrency}, highlight the importance of concurrent
partnerships, with exponential like increase in the number of infected individuals and
the growth rate of the HIV epidemic during its initial phase. This result corroborates
that presented by Morris and Kretzschmar \cite{morrism1997,Kretzschmar2000} and suggests
that the HIV pandemic would be short lived in a monogamous society. This supports the
small world networks theory, which states that only a small fraction of random
connections in the network is needed in order for it to increase its efficiency and allow
fast propagation of infectious diseases or information within its bounds.

The homogeneous representation of sexual partners is not well suited for modelling the
spread of STDs. It provides little information about population structure and
increases the overall epidemic. The core group approach, which divides the population
into risk groups according to their level of sexual activity and exposure to the disease,
gives a better understanding of the population. Not everyone has the same risk of
acquiring and passing on HIV infection to new partners due to the heterogeneity in the
sexual behaviour and different individual immune responses to the virus.

Despite the effectiveness of HAART in delaying the development of AIDS among HIV infected
individuals, the development of a preventive HIV vaccine is the best hope of controlling
the HIV pandemic in the long-term. However an eventual vaccine will not be a magic
bullet, it is likely that the first preventive HIV vaccine will be only partially
effective due to genetically distinct subtypes of the HIV virus. Nevertheless the search
continues and even a low 25\% efficacy vaccine can reduce the HIV transmission if
coverage is high, while a vaccine with 75\% efficacy could markedly reduced HIV pandemic
with relatively low coverage (\ref{swnvaccine}).

\newpage
In any case, vaccine intervention must go alongside education and a wide range of
effective prevention programmes. Those who receive the vaccine must understand that their
risk of contracting HIV infection has lessened but has not vanished.

The spread of sexually transmitted diseases and, in particular, HIV results from a
complex network of dynamic social interactions and other factors related to culture,
sexual behaviour, demography, geography and disease characteristics, as well as the
availability, accessibility and delivery of public healthcare. This thesis gives an
overview of the historical development of STDs, highlights the challenges facing the
world to control the HIV/AIDS epidemic, examines traditional epidemiological models and
introduces HIVacSim, a new modelling framework based upon the small world theory. We
conclude with the following statements:
\begin{itemize}
    \item Sexual behaviour changes towards unsafe sex practices and concurrent partnerships
    dramatically increase the size of the sexually transmitted epidemic;

    \item The introduction of the dynamics of selectivity in interactions for the small
    world network models is as important as the network definition itself;

    \item Connectivity alone will not capture the dynamics, variability and uncertainty
    present in human social and sexual behaviour;

    \item Nodal and edge properties must be taken into account in order to quantify both
    the connections and the spread of disease or information through the network;

    \item Prevention has to be viewed in a global context, however actions must be taken
    locally, at different levels of the community. They should target the sources of long
    range links in the network, responsible for the global spread of the epidemics;

    \item A general behaviour model for the description of human sexual contacts, which
    incorporates actor heterogeneity and dependence, is essential for the future social
    network models of epidemiology. In this context, we developed HIVacSim as our
    contribution to this new era of dynamic networks and epidemiological models.
\end{itemize}

\newpage
\section{Further Work}

The development of the HIVacSim model was successful on meeting the aims of the proposed
research given the available time. The project is a starting point for a long-term
development, which will require further improvements of the model in order to incorporate
the outcome of new research on the dynamics of social interactions, HIV transmission,
treatment and preventive HIV vaccine. Although significant progress has been undoubtedly
achieved, it is vital to continue the research on the dynamics of sexual networks,
improve the knowledge about the rampant HIV pandemic facing the world and add these new
techniques to the HIVacSim model. The following is a list of desired features that should
be carried forward in further development of HIVacSim model:
\parskip=0pt
\begin{itemize}
    \item \emph{Migration} - provides an important factor to the spread of HIV worldwide.
    It includes people travelling for professional reasons (truck drivers, traders,
    military, airline personnel, etc) as well as those running away from their home
    countries, fleeing persecution because of their beliefs, political activism, or wars.
    However, being a migrant, in and of itself, is not a risk factor for HIV. It is the
    situations encountered and the activities undertaken during the migration process that
    are the risk factors. HIVacSim provides external contacts between core group
    populations, but does not allow physical migration of individuals to take place
    between core groups. This is a potential feature to be added to the model, but let
    us not forget that migration is a complex subject in its own right;

    \item \emph{Treatment} - in the absence of an effective HIV vaccine, the availability
    of free HAART treatment has become high in the agenda of governments around the world.
    At present, HIVacSim only accounts for vaccine interventions. Adding support
    for HAART would enable the model to evaluate both preventive and treatment interventions;

    \item \emph{Infectiousness} - the rate of HIV transmission varies by types of exposure
    as well as other facts such as viral load and CD4 cell count \cite{Lifson1989,Lyles2000}. The definition of
    infectious disease within HIVacSim only accounts for types of exposure, therefore more
    flexibility settings should be added in order to represent the effects of treatment
    interventions and accommodate changes to the natural history of the HIV infection;

    \item \emph{HIV subtypes} - it is not clear at the time of this writing if the
    different HIV strains could be dealt with a single vaccine, or if subtype-specific
    vaccines will be needed. The fact however, is that most the vaccines currently under
    development are targeting strains commonly found in North America and Europe. In this
    light, HIV subtypes should be incorporated into HIVacSim;

    \item \emph{Software} - although the current implementation of HIVacSim is very
    flexible from a developer's point of view, it does not provide a friendly GUI to enable
    non computer literate people to use it without expert assistance. The design and
    implementation of a GUI for HIVacSim in order to improve its usability has high
    priority for future development. From a developer's perspective, HIVacSim should be
    made thread safe in order to enable distributed and parallel execution of different
    scenarios, thereby taking advantage of multi-processor computers.
\end{itemize}
\parskip=\baselineskip

There is a great need worldwide to promote the evolved framework and developed model
described within this thesis to a wider healthcare audience. The structure and
flexibility of this model should contribute towards a better understanding of the effects
of social network interactions, sexual behaviour changes and concurrency on the dynamics
of HIV transmission. This would help improve the planning and management of resources to
prevent HIV infection in the first place and make life for those HIV infected individuals
more human and comfortable.
