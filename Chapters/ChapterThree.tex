\chapter{Materiais e Métodos}
\label{chap:mat}


\section{Estrutura analítica do protótipo}
\label{ssec:pbs}
De forma sistemática o projeto ELIR foi dividido em 6 subsistemas, caraterizando a estrutura analítica do projeto (Figura \ref{fig:pbselir}) em:

\begin{itemize}
	\item \textit{perception}
	\item \textit{actuation}
	\item \textit{power management}
	\item \textit{processing system}
	\item \textit{detection}
	\item \textit{software}
\end{itemize}

Os cinco iniciais subsistemas estão mais ligados as condições físicas de hardware do que o subsistema \textbf{software}, porém as implicações de software também devem ser consideradas em cada um destes susbsistemas.

O susbistema \textbf{software} foi idealizado a partir das funcionalidades desenvolvidas  para a solução final do projeto, e estão descriminadas na secção \ref{sec:espf}.

%----- figure --------------------------------------
\begin{figure}[!htb]
	\centering
	\includegraphics[scale=0.6]{Figures/pbselir.png}
	\caption{Esturutura analítica do sistema.}
	%\legend{Fonte: os autores}
	\label{fig:pbselir}
\end{figure}
%---------------------------------------------------

%fazer comentário sobre os subsistemas

A relação final dos componentes do sistema robótico encontra-se no Apêndice \ref{Append:bom}.

%--------- NEW SECTION ----------------------
\section{Especificação dos componentes}
\label{sec:espc}
asjdflkdjsaf




\subsection{Lista de componentes}
\label{ssec:list}
asfkjdsahfkjs


%--------- NEW SECTION ----------------------
%\section{Diagramas mecânicos}
%\label{sec:diagm}



%--------- NEW SECTION ----------------------
\section{Modelo esquemático de alimentação e comunicação}
\label{sec:modesq}



\subsection{Diagramas elétricos}
\label{sec:diage}

\subsection{Esquemas eletrônicos}
\label{ssec:esqe}


%--------- NEW SECTION ----------------------
\section{Especificação das funcionalidades}
\label{sec:espf}
Diante da arquitetura apresentada anteriormente e focando nos objetivos traçados no Capítulo \ref{chap:introd}, o sistema robótico foi dimensionado para onze funcionalidades distintas:

\begin{enumerate}%[itemsep=1pt]
	%\setlength\itemsep{1em}
	\item sistema de verificação da integridade
	\item gerenciamento de energia
	\item aquisição
	\item localização
	\item planejamento de movimento
	\item atuação
	\item detecção
	\item classificação
	\item interface do usuário
	\item autonomia
	\item simulação
\end{enumerate}

A Figura \ref{img:elirfluxo} apresenta o fluxo de informações entre as funcionalidades. Este fluxo deve ser compreendido para que seja estabelecida as relações entre as funcionalidades e o entendimento entre elas, essa compreensão impactará na melhor elaboração da árvore de falhas do sistema e proporcionará um sistema mais confiável.

%---------------picture------------------------------------
\begin{figure} [h!]	
	\caption{Fluxo de informações do sistema.}
	\label{img:elirfluxo}											 
	\centering													 
	\includegraphics[width=1.0\textwidth]{Figures/flxinfofunctionalities}
	%\fautor			 											 
\end{figure}													 
%----------------------------------------------------------

Nas seções seguintes são apresentados em maiores detalhes sobre cada uma das funcionalidades do sistema robótico. Para que fosse melhor compreendido, o desenvolvimento destas funcionalidades foram agrupadas em cinco áreas: movimentação, percepção, interface do usuário, autonomia e simulação. As duas áreas iniciais foram subdivididas em planejamento de movimento, sistema de verificação de integridade, atuação e gerenciamento de energia para a primeira área de nome \textbf{movimentação}, que tem como principal objetivo garantir a execução da missão e transposição de obstáculos. Para a segunda área, nominada por percepção, a subdivisão ficou da seguinte forma: aquisição, detecção, classificação e localização, que como o significado do próprio nome apresenta como objetivo principal a percepção do robô diante do ambiente inserido.

%\subsection{Fluxo das informações}
%\label{ssec:fluxo}



\subsection{Motion Planning}
\label{ssec:motion}
%\subsubsection{Definição da funcionalidade}
A funcionalidade de \textit{Motion Planning} é responsável por realizar o planejamento da trajetória do Robô, utilizando o software \textit{MoveIt!} que realiza o cálculo da cinemática inversa para encontrar a melhor forma de ultrapassar os obstáculos.
\subsubsection{Dependências}
O software moveit pode utilizar o modelo matemático da cinemática inversa do robô ou um arquivo do tipo URDF.
O nome URDF é uma sigla para \textit{Unified Robot Description Format}, esse arquivo é uma especificação em \verb|XML| utilizada para descrever robôs. Modelos em URDF apresentam uma simplicidade na descrição do robô, e para o caso do Robô \textit{Elir}, utilizar o modelo URDF possibilitará uma aproximação fiel ao modelo real do robô, assim para o cálculo da cinemática inversa será utilizado o seu modelo URDF e não o seu modelo matemático.

\subsubsection{Premissas Necessárias}
Para o correto funcionamento dessa funcionalidade as seguintes premissas são necessárias:
\begin{itemize}
	\item A configuração dos limites de giro das juntas do robô estarão compatíveis com os comandos enviados
	\item O modelo URDF do robô estará adequado com o modelo físico
	\item O pacote gerado pelo \textit{MoveIt! Setup Assistant} estará configurado adequadamente
\end{itemize}
\subsubsection{Descrição da Funcionalidade}
A movimentação do robô na linha acontecerá por movimentos de translação e transposição de obstáculos. A translação na linha será feita por controladores de torque nas rodas do robô, enquanto a transposição do obstáculos utilizará o moveit.
Por meio da ferramenta \textit{MoveIt! Setup Assistant}, se utiliza o modelo do robô para criar um pacote do ROS com os principais arquivos pelo moveit. 
A configuração correta do moveit possibilita que se utilizem as funções da sua biblioteca para o cálculo da trajetória, levando em consideração também obstáculos no caminho.

O moveit fornece uma \textit{user interface} que recebe o end-effector,a nomenclatura atribuída ao node feito em python que recebe o \textit{end-effector} é \verb|moveit_commander|. O  \textit{node} responsável por fazer a integração da user interface com os parâmetros recebidos pelo \textit{ROS Parameter Server} com o \textit{end-effector} para fazer os cálculos é denominado \verb|move_group|. O \textit{node} \verb|move_group| também pode receber parâmetros como leituras dos sensores do robô e nuvens de pontos.

\begin{figure}[!h]
	\centering
	\includegraphics[width=0.8\textwidth]{motion_plan_func.png}
	\caption{Fluxograma de funcionamento da funcionalidade de Motion Planning}
	\label{fig:flux_motion}
	\source{Própria}
\end{figure}

\subsubsection{Saídas}
Por meio da compatibilização do \textit{MoveIt!} com o \textit{ROS}, a saída dessa funcionalidade são os comandos de velocidade, esforço e posição para cada junta do robô.

\subsection{Actuation }
\label{ssec:actu}
\subsubsection{Definição da funcionalidade}
A funcionalidade de Actuation tem como objetivo mover a estrutura física do robô, possibilitando o controle dos movimentos das juntas, garras e unidades de tração.
\subsubsection{Dependências}
Essa funcionalidade depende das funcionalidades de \textit{Power Management} e \textit{Motion Planning}. O \textit{Power Management} será responsável por fazer alimentação dos motores, possibilitando controlar a corrente máxima fornecida para cada grupo.
A dependência em relação à funcionalidade de \textit{Motion Planning} está atrelada principalmente com o software \textit{MoveIt!}, que ao receber um \textit{end-effector},realiza o cálculo de trajetória e envia os comandos de velocidade, esforço e posição para os controladores das juntas, garras e unidades de tração.

\subsubsection{Premissas Necessárias}
Para o correto funcionamento desse módulo, devem ser consideradas as seguintes premissas:
\begin{itemize}
	\item Os motores devem estar configurados de acordo com o padrão de ID determinado pela equipe, fazendo parte da mesma malha de controle;
	\item Os controladores das juntas,garras e unidades devem estar configurados de acordo com os comandos que serão recebidos pelo MoveIt!;
	\item Os 3 grupos de motores estarão em malhas de alimentação de 12V individuais.
\end{itemize}
\subsubsection{Descrição da Funcionalidade}
O ROS disponibiliza uma série de drivers para compatibilização dos motores dynamixel, possibilitando a criação de controladores específicos no seu ambiente. Serão criados os controladores referentes as juntas e unidades de tração do robô.Os controladores receberão comandos de \textit{velocity} e \textit{position} do \textit{MoveIt!} junto com os comandos para movimentar o robô na linha.
Após os comandos serem recebidos pelos controladores, eles serão enviados para o \textit{hardware} do robô, de acordo do padrão de comunicação dos motores, por meio de comunicação serial. 
\begin{figure}[h]
	\centering
	\includegraphics[width=0.6\textwidth]{actuation_depen.png}
	\caption{Fluxograma da funcionalidade Actuation}
	\label{fig:depen_actuation}
	\source{Própria}
\end{figure}
\subsubsection{Saídas}
A saída desta funcionalidade é o movimento da estrutura física do robô, que estará de acordo com o planejamento de trajetória do \textit{MoveIt!} e com as instruções para operação na linha

\subsection{Power Management}
\label{ssec:power}
\subsubsection{Definição da funcionalidade}

A funcionalidade de \textit{Power Management} é responsável por administrar o fornecimento de energia para os dispositivos eletrônicos do robô, nos níveis adequados de tensão e corrente.

\subsubsection{Dependências}
Essa funcionalidade depende da comunicação serial por meio da biblioteca \textit{rosserial} e da operacionalização do firmware embarcado no hardware (placa) de acordo com as necessidades do projeto.

\subsubsection{Premissas Necessárias}
Para o correto funcionamento desse módulo de \textit{Power Management}, devem ser consideradas as seguintes premissas:
\begin{itemize}
	\item A placa multiplexadora estará conectada diretamente ao módulo de \textit{Power Management} 
	\item Todos os dispositivos estarão conectados nas suas respectivas entradas
	\item A placa deverá ser alimentada por 2 baterias
	\item A placa estará conectada diretamente na NUC, por meio de uma USB	
\end{itemize}
\begin{figure}[h]
	\centering
	\includegraphics[width=1\textwidth]{power_management_hardware.png}
	\caption{Fluxograma de funcionamento da funcionalidade de Power Management}
	\label{fig:power_management_hardware}
	\source{Própria}
\end{figure}
\subsubsection{Descrição da Funcionalidade}
A placa de \textit{Power Management} fornece diversos recursos para integração com o ROS. Seu firmware, além de realizar as medições e controle dos níveis de tensão e corrente para alimentação do robô, estará adaptado com as seguintes funcionalidades para que haja integração do hardware com o ROS:
\begin{itemize}
	\item \textit{Publishers} que contém os status das portas em níveis de tensão e corrente; avisos de surtos de corrente ou sobre-corrente; disponibilidade do hardware de \textit{Power Management}
	\item \textit{Serviços} para realizar a verificação dos níveis de corrente; definição dos limites de corrente nas portas; realização de comandos on-off
\end{itemize}
O conjunto de baterias fornecerá a energia para o sistema, a placa de \textit{Power Management} irá administrar a distribuição da energia para os seguintes componentes:
\begin{itemize}
	\item Grupos de servo motores
	\item Grupo de sensores de corrente
	\item NUC
	\item Interface HUB
	\item Câmera LWIR
	\item Sensor ultrassônico
\end{itemize}

\subsubsection{Saídas}
A funcionalidade irá disponibilizar a energia para o robô e as seguintes estruturas no ambiente ROS:
\begin{itemize}
	\item Tópicos com informações de tensão e corrente nas portas
	\item Tópico para aviso de sobre-corrente
	\item Tópico para informar disponibilidade da placa
	\item Serviços para ler e configurar limite de corrente das portas
	\item Serviço para ligar ou desligar energia em uma porta	
\end{itemize}

\subsection{System Integrity Check}
\label{ssec:check}

\subsubsection{Definição da funcionalidade}
É a funcionalidade responsável por checar a integridade do sistema antes do início da missão, verificando os subsistemas e suas variáveis.

\subsubsection{Dependências}
A funcionalidade receberá informações dos seguintes componentes
\begin{itemize}
	\item Sensor de Temperatura
	\item Servomotores
	\item Câmera IR
	\item Câmera Stéreo
	\item IMU
	\item Sensor de Proximidade
	\item Placa de Power Management
	\item Sonar 
	\item Baterias
\end{itemize}

Todas as informações serão enviadas por meio do ambiente ROS, na forma de \textit{Services} ou \textit{Publishers}.

\subsubsection{Premissas Necessárias}
As premissas necessárias para o funcionamento dessa funcionalidade são:
\begin{itemize}
	\item Os subsistemas do robô irão disponibilizar o seu status no ambiente ROS por meio de tópicos ou serviços
	\item A checagem fará parte do planejamento de missão
\end{itemize}

\subsubsection{Descrição da Funcionalidade}
A checagem da integridade do sistema é uma funcionalidade essencial para garantir o sucesso da missão e preservar a integridade do robô. O ROS facilita essa comunicação entre os subsistemas, possibilitando que seja criada uma rotina de checagem antes de cada missão.

Será disponibilizado no sistema uma rotina para iniciar a missão. Ao receber o comando para início de missão, os sistemas serão checados sequencialmente, utilizando estrutura de \textit{Services} e \textit{Publishers} do ROS. Caso algum sistema apresente falha, a missão não se iniciará e o erro será mostrado no \textit{terminal} e registrado no arquivo de \verb|log|. Se todos os sistemas estiverem em funcionamento, se iniciará a missão. O fluxograma da funcionalidade está ilustrado na figura \ref{fig:sys_check_flux}.	
\begin{figure}[h]
	\centering
	\includegraphics[width=0.6\textwidth]{sys_check_flux.png}
	\caption{Fluxograma da rotina para checagem do sistema}
	\label{fig:sys_check_flux}
	\source{Própria}
\end{figure} 


\subsubsection{Saídas}
No início da rotina de inspeção, a funcionalidade será responsável por enviar o sinal inicia a missão. Caso todos os sistemas checados estejam funcionando, a inspeção ocorrerá normalmente, se algum sistema apresentar defeitos, o defeito será mostrado no \textit{terminal}, registrado em log e a missão será abortada.
%--------- NEW SECTION ----------------------
%\section{Interface do Usuário}
%\label{sec:ui}
%Este item será abordado pelo time Perception


%--------- NEW SECTION ----------------------
\section{Simulação do sistema}
\label{sec:sim}




