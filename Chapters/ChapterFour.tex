\chapter{Resultados}
\label{chap:result}
asdfdsfdsf

%--------- NEW SECTION ----------------------
\section{Testes unitários}
\label{sec:testu}
asdfadsfsdfs



%--------- NEW SECTION ----------------------
\section{Testes integrados}
\label{sec:testi}
asdfadsfsdfs

%--------- NEW SECTION ----------------------
\section{Avaliação da prontidão tecnológica}
\label{sec:trl}
Uma das ferramentas conhecidas para a avaliação de tecnologias é a matriz TRL desenvolvida pela NASA\footnote{National Aeronautics and Space Administration} e que permitir definir o nível de maturidade de uma tecnologia. Entende-se que a tecnologia é definida como a aplicação prática do conhecimento para criar a capacidade de fazer algo inteiramente novo de forma inteiramente nova, o que difere da pesquisa científica que engloba a descoberta de um novo conhecimento da qual a tecnologia é derivada.

A importância do uso dessa avaliação, encontra seu respaldo no quesito de nortear o desenvolvimento do projeto na minimização dos gastos oriundos de uma orçamentação e também no conhecimento das tecnologias plausíveis para o desenvolvimento da solução requerida.

%\section{A matriz BTRL}
Todo projeto foi acompanhado por uma cadência de avaliações ao longo das fases do projeto. Estas avaliações seguem as diretrizes da ISO 16290, a qual estabelece níveis do quanto uma tecnologia está desenvolvida, tomando como base este procedimento e levando em consideração as avaliações de risco para um projeto de R\&D\footnote{Research and Development}, criou-se o BTRL\footnote{BIR Technology Readiness Level}. 

% Please add the following required packages to your document preamble:
% \usepackage[table,xcdraw]{xcolor}
% If you use beamer only pass "xcolor=table" option, i.e. \documentclass[xcolor=table]{beamer}
%\begin{sidewaystable*}
\begin{table}[h]
\centering
\caption{Matriz do Nível de Prontidão Tecnológia do Senai Cimatec.}
\begin{adjustbox}{width=1\textwidth}
\label{tabela:BTRL}
\begin{tabular}{lc|p{1.5cm} p{1.5cm} p{1.5cm} p{1.5cm}}
\cline{1-2}
\multicolumn{2}{|l|}{\cellcolor[HTML]{000000}{\color[HTML]{FFFFFF} \textbf{NÍVEL DA PRONTIDÃO TECNOLÓGICA}}} &  &  &  &  \\ \cline{1-2}
\multicolumn{1}{|l|}{\textbf{perspectiva}} & \textbf{nível} &  &  &  &  \\ \hline
\multicolumn{1}{|l|}{\textit{sistema aprovado}} & \textit{9} & \multicolumn{1}{c|}{\cellcolor[HTML]{F56B00}L2} & \multicolumn{1}{c|}{\cellcolor[HTML]{F8FF00}{\color[HTML]{FFFFFF} L3}} & \multicolumn{1}{c|}{\cellcolor[HTML]{009901}{\color[HTML]{FFFFFF} L4}} & \multicolumn{1}{c|}{\cellcolor[HTML]{009901}{\color[HTML]{FFFFFF} L4}} \\ \hline
\multicolumn{1}{|l|}{\textit{sistema qualificado}} & \textit{8} & \multicolumn{1}{c|}{\cellcolor[HTML]{F56B00}L2} & \multicolumn{1}{c|}{\cellcolor[HTML]{F8FF00}{\color[HTML]{FFFFFF} L3}} & \multicolumn{1}{c|}{\cellcolor[HTML]{009901}{\color[HTML]{FFFFFF} L4}} & \multicolumn{1}{c|}{\cellcolor[HTML]{009901}{\color[HTML]{FFFFFF} L4}} \\ \hline
\multicolumn{1}{|l|}{\textit{protótipo testado em campo operacional}} & \textit{7} & \multicolumn{1}{c|}{\cellcolor[HTML]{F56B00}L2} & \multicolumn{1}{c|}{\cellcolor[HTML]{F8FF00}{\color[HTML]{FFFFFF} L3}} & \multicolumn{1}{c|}{\cellcolor[HTML]{009901}{\color[HTML]{FFFFFF} L4}} & \multicolumn{1}{c|}{\cellcolor[HTML]{009901}{\color[HTML]{FFFFFF} L4}} \\ \hline
\multicolumn{1}{|l|}{\textit{protótipo testado em campo relevante}} & \textit{6} & \multicolumn{1}{c|}{\cellcolor[HTML]{9A0000}{\color[HTML]{FFFFFF} L1}} & \multicolumn{1}{c|}{\cellcolor[HTML]{F56B00}L2} & \multicolumn{1}{c|}{\cellcolor[HTML]{F8FF00}{\color[HTML]{FFFFFF} L3}} & \multicolumn{1}{c|}{\cellcolor[HTML]{009901}{\color[HTML]{FFFFFF} L4}} \\ \hline
\multicolumn{1}{|l|}{\textit{funcionalidades testadas em campo relevante}} & \textit{5} & \multicolumn{1}{c|}{\cellcolor[HTML]{9A0000}{\color[HTML]{FFFFFF} L1}} & \multicolumn{1}{c|}{\cellcolor[HTML]{F56B00}L2} & \multicolumn{1}{c|}{\cellcolor[HTML]{F8FF00}{\color[HTML]{FFFFFF} L3}} & \multicolumn{1}{c|}{\cellcolor[HTML]{009901}{\color[HTML]{FFFFFF} L4}} \\ \hline
\multicolumn{1}{|l|}{\textit{funcionalidades testadas em laboratório}} & \textit{4} & \multicolumn{1}{c|}{\cellcolor[HTML]{9A0000}{\color[HTML]{FFFFFF} L1}} & \multicolumn{1}{c|}{\cellcolor[HTML]{F56B00}L2} & \multicolumn{1}{c|}{\cellcolor[HTML]{F56B00}L2} & \multicolumn{1}{c|}{\cellcolor[HTML]{F8FF00}{\color[HTML]{FFFFFF} L3}} \\ \hline
\multicolumn{1}{|l|}{\textit{conceito aprovado}} & \textit{3} & \multicolumn{1}{c|}{\cellcolor[HTML]{9A0000}{\color[HTML]{FFFFFF} L1}} & \multicolumn{1}{c|}{\cellcolor[HTML]{9A0000}{\color[HTML]{FFFFFF} L1}} & \multicolumn{1}{c|}{\cellcolor[HTML]{F56B00}L2} & \multicolumn{1}{c|}{\cellcolor[HTML]{F56B00}L2} \\ \hline
\multicolumn{1}{|l|}{\textit{conceito formulado}} & \textit{2} & \multicolumn{1}{c|}{\cellcolor[HTML]{9A0000}{\color[HTML]{FFFFFF} L1}} & \multicolumn{1}{c|}{\cellcolor[HTML]{9A0000}{\color[HTML]{FFFFFF} L1}} & \multicolumn{1}{c|}{\cellcolor[HTML]{9A0000}{\color[HTML]{FFFFFF} L1}} & \multicolumn{1}{c|}{\cellcolor[HTML]{F56B00}L2} \\ \hline
\multicolumn{1}{|l|}{\textit{princípios básicos aprovados}} & \textit{1} & \multicolumn{1}{c|}{\cellcolor[HTML]{9A0000}{\color[HTML]{FFFFFF} L1}} & \multicolumn{1}{c|}{\cellcolor[HTML]{9A0000}{\color[HTML]{FFFFFF} L1}} & \multicolumn{1}{c|}{\cellcolor[HTML]{9A0000}{\color[HTML]{FFFFFF} L1}} & \multicolumn{1}{c|}{\cellcolor[HTML]{9A0000}{\color[HTML]{FFFFFF} L1}} \\ \hline
 &  & \multicolumn{1}{c|}{\textbf{\begin{tabular}[c]{@{}c@{}}muito \\ alto\end{tabular}}} & \multicolumn{1}{c|}{\textbf{alto}} & \multicolumn{1}{c|}{\textbf{médio}} & \multicolumn{1}{c|}{\textbf{baixo}} \\ \cline{3-6} 
 &  & \multicolumn{4}{c|}{\cellcolor[HTML]{000000}{\color[HTML]{FFFFFF} \textbf{CATEGORIZAÇÃO DO RISCO TÉCNICO}}} \\ \cline{3-6} 
\end{tabular}
\end{adjustbox}
\end{table}	
%\end{sidewaystable*}

O BTRL é o índice destas duas variáveis: TRL e Riscos. A matriz apresenta através da Tabela \ref{tabela:BTRL} de forma clara a relação entre estes dois critérios, estabelecendo desta forma 4 níveis:
\begin{itemize}
	\item L1 - desenvolvimento significativo requerido.
	\item L2 - requer mais desenvolvimento da tecnologia antes de prosseguir para a próxima fase do projeto.
	\item L3 - requer alguns desenvolvimento adicional na tecnologia.
	\item L4 - pronto para uso tanto do protótipo para as fases seguintes, como para o uso do produto em campo.
\end{itemize}
No entanto, deve-se estabelecer critérios para os níveis de riscos atingidos. Onde foi utilizado para a avaliação destes riscos outras duas variáveis importantes: o nível de confiabilidade do sistema e o nível do protótipo/sistema que se está sendo desenvolvido (conforme Tabela \ref{tabela:TRC}).

% Please add the following required packages to your document preamble:
% \usepackage[table,xcdraw]{xcolor}
% If you use beamer only pass "xcolor=table" option, i.e. \documentclass[xcolor=table]{beamer}
\begin{table}[h]
\centering
\caption{Categorização dos Riscos Técnicos - TRC.}
\begin{adjustbox}{width=1\textwidth}
\label{tabela:TRC}
\begin{tabular}{lc|p{2.5cm}p{2.5cm}p{2.5cm}p{2.5cm}}
\cline{1-2}
\multicolumn{2}{|l|}{\cellcolor[HTML]{000000}{\color[HTML]{FFFFFF} \textbf{PROTÓTIPO}}} &  &  &  &  \\ \hline
\multicolumn{1}{|l|}{tecnologia aprovada} & 4 & \multicolumn{1}{c|}{\cellcolor[HTML]{F56B00}{\color[HTML]{FFFFFF} B}} & \multicolumn{1}{c|}{\cellcolor[HTML]{F8FF00}C} & \multicolumn{1}{c|}{\cellcolor[HTML]{009901}{\color[HTML]{FFFFFF} D}} & \multicolumn{1}{c|}{\cellcolor[HTML]{009901}{\color[HTML]{FFFFFF} D}} \\ \hline
\multicolumn{1}{|l|}{pequenas modificações} & 3 & \multicolumn{1}{c|}{\cellcolor[HTML]{F56B00}{\color[HTML]{FFFFFF} B}} & \multicolumn{1}{c|}{\cellcolor[HTML]{F8FF00}C} & \multicolumn{1}{c|}{\cellcolor[HTML]{F8FF00}C} & \multicolumn{1}{c|}{\cellcolor[HTML]{009901}{\color[HTML]{FFFFFF} D}} \\ \hline
\multicolumn{1}{|l|}{grandes modificações} & 2 & \multicolumn{1}{c|}{\cellcolor[HTML]{9A0000}{\color[HTML]{FFFFFF} A}} & \multicolumn{1}{c|}{\cellcolor[HTML]{F56B00}{\color[HTML]{FFFFFF} B}} & \multicolumn{1}{c|}{\cellcolor[HTML]{F8FF00}C} & \multicolumn{1}{c|}{\cellcolor[HTML]{F8FF00}C} \\ \hline
\multicolumn{1}{|l|}{novo design conceitual} & 1 & \multicolumn{1}{c|}{\cellcolor[HTML]{9A0000}{\color[HTML]{FFFFFF} A}} & \multicolumn{1}{c|}{\cellcolor[HTML]{9A0000}{\color[HTML]{FFFFFF} A}} & \multicolumn{1}{c|}{\cellcolor[HTML]{F56B00}{\color[HTML]{FFFFFF} B}} & \multicolumn{1}{c|}{\cellcolor[HTML]{F56B00}{\color[HTML]{FFFFFF} B}} \\ \hline
 &  & \multicolumn{1}{p{2.5cm}|}{\footnotesize{melhorias na tecnologia são requeridas}} & \multicolumn{1}{p{2.5cm}|}{\footnotesize{melhorias no design são requeridas}} & \multicolumn{1}{p{2.5cm}|}{\footnotesize{pequenas melhorias são requeridas}} & \multicolumn{1}{p{2.5cm}|}{\footnotesize{não há necessidade de melhorias}} \\ \cline{3-6} 
 &  & \multicolumn{4}{c|}{\cellcolor[HTML]{000000}{\color[HTML]{FFFFFF} \textbf{CONFIABILIDADE}}} \\ \cline{3-6} 
\end{tabular}
\end{adjustbox}
\end{table}

A avaliação dos riscos técnicos (TRC) é categorizada em 4 níveis:
\begin{itemize}
	\item A - risco muito alto
	\item B - risco alto
	\item C - risco médio
	\item D - risco baixo
\end{itemize}

Para os projetos em robótica, será sempre avaliado o protótipo/sistema no final de cada fase do desenvolvimento. Neste caso específico do projeto de Direção Assistida, está sendo avaliado o sistema na fase Conceitual.

\subsection{Avaliação do BTRL para o sistema robótico em desenvolvimento}
De forma a sistematizar a avaliação dos subsistemas, em desenvolvimento para a avaliação do nível de prontidão tecnológica nesta fase do projeto foi tomado a estrutura da arquitetura geral apresentada no início do capítulo \ref{ch:arqgen}.
Desta forma obteve-se a seguinte avaliação, conforme apresentada na Tabela \ref{tabela:TRL_FC}.

% Please add the following required packages to your document preamble:
% \usepackage[table,xcdraw]{xcolor}
% If you use beamer only pass "xcolor=table" option, i.e. \documentclass[xcolor=table]{beamer}
\begin{table}[h]
%\scalefont{0.8}
\centering
\caption{Avaliação da Prontidão Tecnológica - Fase Design.}
\label{tabela:TRL_FC}
\begin{tabular}{llccccl}
\rowcolor[HTML]{000000} 
\multicolumn{7}{l}{\cellcolor[HTML]{000000}{\color[HTML]{FFFFFF} FASE DESIGN}} \\
\rowcolor[HTML]{EFEFEF} 
\textbf{subsistemas} 		& \textbf{componentes} 				& RL & PL & TRC & TRL & BRTL\\ \hline
sensoriamento				& gps								& 4	 & 4  & 4	& 9	  & \cellcolor[HTML]{009901}\\ \hline
sensoriamento				& imu								& 4	 & 4  & 4	& 9	  & \cellcolor[HTML]{009901}\\ \hline
sensoriamento				& rgb cam							& 4	 & 4  & 4	& 9	  & \cellcolor[HTML]{009901}\\ \hline
sensoriamento				& vnir cam							& 3	 & 3  & 3	& 8	  & \cellcolor[HTML]{009901}\\ \hline
sensoriamento				& swir cam							& 3	 & 3  & 3	& 8	  & \cellcolor[HTML]{009901}\\ \hline
sensoriamento				& lidar								& 4	 & 4  & 4	& 9	  & \cellcolor[HTML]{009901}\\ \hline
sistema de processamento	    & dau								& 4	 & 4  & 4	& 9	  & \cellcolor[HTML]{009901}\\ \hline
sistema de processamento 	& controlador XY						& 3	 & 4  & 4	& 9	  & \cellcolor[HTML]{009901}\\ \hline
sistema de processamento 	& unidade de rotação					& 3	 & 4  & 4	& 9	  & \cellcolor[HTML]{009901}\\ \hline
sistema de processamento 	& switch								& 4	 & 4  & 4	& 9	  & \cellcolor[HTML]{009901}\\ \hline
sistema de processamento 	& NUC								& 4	 & 4  & 4	& 9	  & \cellcolor[HTML]{009901}\\ \hline
sistem de potência			& baterias							& 4	 & 4  & 4	& 9	  & \cellcolor[HTML]{009901}\\ \hline
sistem de potência			& fonte de alimentação				& 4	 & 4  & 4	& 9	  & \cellcolor[HTML]{009901}\\ \hline
sistem de potência			& no break							& 4	 & 4  & 4	& 9   & \cellcolor[HTML]{009901}\\ \hline
estrutura mecânica			& carenagem							& 4	 & 4  & 4	& 9	  & \cellcolor[HTML]{009901}\\ \hline
estrutura mecânica			& suporte dos sensores				& 4	 & 4  & 4	& 9	  & \cellcolor[HTML]{009901}\\ \hline
estrutura mecânica			& perfis de alumínio					& 4	 & 4  & 4	& 9	  & \cellcolor[HTML]{009901}\\ \hline
estrutura mecânica			& rodas								& 4	 & 4  & 4	& 9	  & \cellcolor[HTML]{009901}\\ \hline
interface					& base de controle					& 4	 & 4  & 4	& 9	  & \cellcolor[HTML]{009901}\\ \hline
interface					& display							& 4	 & 4  & 4	& 9	  & \cellcolor[HTML]{009901}\\ \hline
funcinalidades				& aquisição							& 3	 & 3  & 3	& 5	  & \cellcolor[HTML]{F8FF00}\\ \hline
funcinalidades				& calibração							& 2	 & 2  & 2	& 4	  & \cellcolor[HTML]{F56B00}\\ \hline
funcinalidades				& checking							& 2	 & 2  & 2	& 4	  & \cellcolor[HTML]{F56B00}\\ \hline
funcinalidades				& localização						& 2	 & 3  & 3	& 5	  & \cellcolor[HTML]{F8FF00}\\ \hline
funcinalidades				& navegação							& 2	 & 3  & 3	& 5	  & \cellcolor[HTML]{F8FF00}\\ \hline
funcinalidades				& escaneamento						& 2	 & 2  & 2	& 4	  & \cellcolor[HTML]{F56B00}\\ \hline
funcinalidades				& gestão de dados					& 3	 & 3  & 3	& 4	  & \cellcolor[HTML]{F56B00}\\ \hline
funcinalidades				& gestão de log						& 3	 & 3  & 3   & 4	  & \cellcolor[HTML]{F56B00}\\ \hline
funcinalidades				& interface de operação				& 2	 & 2  & 2	& 4	  & \cellcolor[HTML]{F56B00}\\ \hline
funcinalidades				& interface de pós-processamento		& 2	 & 2  & 2	& 2	  & \cellcolor[HTML]{9A0000}\\ \hline
funcinalidades				& pós-processamento					& 2	 & 2  & 2	& 3	  & \cellcolor[HTML]{9A0000}\\ \hline
\end{tabular}
\end{table}

%Diante disso, faz-se os seguintes comentários:
%\begin{itemize}
%	\item \textbf{VNIR cam}: o conjunto de câmeras hiperespectrais foram testadas em laboratório, quanto a sua operação e funcionamento, as mesmas apresentaram um bom desempenho mas precisam ser integradas ao sistema; seu desempenho em ambiente externo mostrou-se adequado.
%	\item \textbf{SWIR cam}: o conjunto de câmeras hiperespectrais foram testadas em laboratório, quanto a sua operação e funcionamento, as mesmas apresentaram um bom desempenho mas precisam ser integradas ao sistema; seu desempenho em ambiente externo mostrou-se adequado.
%	\item \textbf{aquisição}: esta funcionalidade testada e com dados disponibilizados para testes e consultas.
%	\item \textbf{calibração}: a funcionalidade apresenta em grande parte a integração com os sensores, testes precisam ser realizados para uma melhor otimização.
%	\item \textbf{checking}: funcionalidade isolada no início do desenvolvimento, aumentar atenção para uma aplicação mais profunda.
%	\item \textbf{localização}: intermitência no uso, novas técnicas precisam ser testadas para finalizar o desenvolvimento.
%	\item \textbf{navegação}: funcionalidade em adaptação pela falta da plataforma móvel.
%	\item \textbf{escaneamento}: testes elaborados em laboratório apresentaram uma boa eficiência, precisa ser testado em ambiente relevante.
%	\item \textbf{gestão de dados}: funcionalidade desenvolvida sob o framework ROS, necessita maior entendimento entre a troca de informações com o sistema operacional Windows.
%	\item \textbf{gestão de log}: algoritmo em otimização, muito da funcionalidade já testada em outros projetos.
%	\item \textbf{interface de operação}: apresenta-se de forma simplificada e útil para o contexto de prototipagem.
%	\item \textbf{interface de pós-processamento}: necessita desenvolver, a ideia ainda esta baseada em um \textit{draft}.
%	\item \textbf{pós-processamento}: funcionalidade ainda não testada, porém apresenta consistência na elaboração do algoritmo.
%%	
%\end{itemize}

Tendo como base do desenvolvimento os requisitos técnicos levantados, deve-se também observar os comentários levantados pela avaliação do BTRL e potencializar o desenvolvimento para atingir o nível 4 ou 3 do BTRL dependendo da tecnologia envolvida; acredita-se que nestes níveis o protótipo poderá ser considerado adequado para o nível de TRL 5/6 estabelecido como objetivo pela direção do projeto.

%--------- NEW SECTION ----------------------
\section{Trabalhos futuros}
\label{sec:trabfut}
asdfadsfsdfs





