\chapter{Conclusão}
\label{chap:conc}
\begin{flushright}
	"Se vi mais longe foi por estar de pé sobre ombros de gigantes." \\
	\ \\
	(Sir Isaac Newton)
\end{flushright}

Neste trabalho foi desenvolvido o sistema de movimentação para um protótipo de robô de inspeção de linha. O fluxo metodológico foi uma grande conquista desse projeto, onde o se buscou seguir as diretrizes de desenvolvimento de projetos robóticos de alto porte, utilizando ferramentas de referência para estabelecer um conceito concreto, e também produzindo diversos tipos de documentação que possibilitaram a produção de um conteúdo palpável em relação a esse trabalho.

O desenvolvimento de aplicações que foi realizado para o projeto, consiste em um conteúdo de suma importância para desenvolvimento de projetos similares, buscou-se utilizar as boas práticas de organização e implementar os padrões utilizados para outros projetos de desenvolvimento. 

Inicialmente o projeto consistia em uma fuga da zona de conforto, apresentando diversos conteúdos novos e solicitando uma nova visão sobre o trabalho com engenharia, e com o seu caminhar, saiu da concepção da idéia e desafio do aprendizado para o desenvolvimento de um sistema robótico real. 

O conceito gerado para o sistema, utiliza parâmetros de operação semelhantes à sistemas de alta complexidade, apresentando um fluxo de informações de alto nível, possibilitando uma noção sobre o funcionamento do sistema e integração com os outros subsistemas já desenvolvidos. 

A organização do sistema, junto com suas ferramentas e conhecimentos produzidos, representa uma grande conquista do projeto. O conhecimento deixado relacionado ao \textit{framework} \textit{ROS}, as conclusões tiradas da ferramenta \textit{MoveIt!},assim como o comportamento observado pelos dispositivos físicos faz com que haja uma referência para projetos semelhantes, possibilitando a criação de conceitos cada vez mais consistentes e execução de idéias de forma efetiva.

A estrutura do protótipo montada, com suas ligações para comunicação e alimentação feitas devidamente, assim como esquemáticos desenvolvidos especialmente para o projeto, possibilita que o mesmo seja tomado como base de estudo, ao se realizar o design e desenvolvimento de outros protótipos.

O uso da estrutura do \textit{ROS} possibilita que seja feita a integração com outros sistemas sem muitos problemas, já que foram adotados os padrões do \textit{framework}, assim possibilitando que o que foi desenvolvido para esse robô, seja utilizado em futuras aplicações, sejam elas tomando como base esse protótipo ou outros semelhantes.

A documentação técnica produzida seguiu de acordo com as orientações de projeto visando o reaproveitamento e continuação do trabalho, atendendo as demandas e solicitações para o desenvolvimento do projeto.

O projeto alcançou as expectativas, problemas inesperados encontrados durante o seu desenvolvimento foram contornados e os grandes desafios do projeto foram concluídos, onde a experiência como um todo foi muito enriquecedora e se deu de forma prazerosa, onde todos os participantes conseguiram aprimorar suas habilidades, crescer como pessoa e como profissional. Pelas palavras do sábio Platão: "A coisa mais indispensável a um homem é reconhecer o uso que deve fazer do seu próprio conhecimento."
	
