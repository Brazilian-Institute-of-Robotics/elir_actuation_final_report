\chapter{Conclusão}
\label{chap:conc}

O resultado do projeto alcançou as expectativas. Os problemas que aconteceram conseguiram ser contornados e o tempo gasto para sua solução conseguiu se adequar ao esperado pelo cronograma das tarefas. O material produzido atende às demandas do cliente e os pacotes produzidos foram organizados buscando facilitar o uso por terceiros.


\section{Considerações finais}
\label{sec:consid}

A gestão do projeto do robô como um todo, fez com que o resultado produzido alcançasse as expectativas. O nível de desenvolvimento aumentou progressivamente de forma que o projeto foi conduzido, adicionando paulatinamente diversos conhecimentos específicos que não seriam vistos normalmentes durante a graduação, mas também fortalecendo conhecimentos já formados.

O que foi produzido para o projeto estará disponível para futuras consultas, o que impulsiona o desenvolvimento de projetos semelhantes. Novos estudos podem ser iniciados como trabalhos de graduação, ou pós-graduação, realizando provas de conceito e abrindo oportunidades para novas tecnologias.

O início do projeto se deu de forma lenta, por apresentar uma área do conhecimento nova para maior parte da equipe. A robótica necessita da integração de diversas áreas diferentes, e para a engenharia elétrica, o conhecimento de diversas camadas de abstração. Com a experiência e o desenvolver das atividades, os conhecimentos adquiridos possibilitaram atividades em paralelo e o aumento da versatilidade dos integrantes.

A experiência como um todo foi muito enriquecedora, adicionando conhecimentos que serão necessários no futuro e proporcionando um crescimento para todos os participantes.
