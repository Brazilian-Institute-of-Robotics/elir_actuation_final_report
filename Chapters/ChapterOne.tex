\chapter{Introdução}
\label{chap:intro}
O Brasil apresenta uma matriz energética diferente da do resto do mundo, onde as fontes renováveis representam uma grande parte da geração da energia. Segundo a \cite{epe_site}, em 2016, a matriz energética mundial contava com somente 14,1\% da matriz energética constituída por fontes renováveis, enquanto o brasil já apresentava 82\% da sua matriz vinda de fontes renováveis, onde a geração hidrelétrica corresponde a 70\% dessa geração.

A expectativa é de que a energia hidrelétrica continue sendo cada vez mais utilizada no país, devido ao crescimento previsto da demanda energética brasileira, onde segundo o \cite{atlas_aneel} o consumo atual é de 405 TWh e a demanda esperada em 2030 é de 950 e 1.250 TWh/ano \cite{bronzatti_matrizes}. Mesmo com a grande participação da geração hidrelétrica , somente 23\% dos 260 GW totais de potencial hidrelétrico são aproveitados \cite{atlas_aneel}.

A concentração de demanda energética no Brasil está concentrada principalmente na região Sudeste devido a densidade populacional e elevada industrialização, isso faz com que dois terços do total da  capacidade instalada estejam localizadas na Bacia do Rio Paraná  que é a bacia mais pŕoxima da região, enquanto as bacias com potencial menos aproveitado são as localizadas no norte e nordeste do país \cite{atlas_aneel}.

Com desenvolvimento do país é esperado um aumento na demanda de energia elétrica e consequentemente um aumento na geração de energia hidrelétrica, isso faz com que seja esperado um crescimento considerável na quantidade das linhas de transmissão, de acordo com \cite{MME}, em setembro de 2018 o sistema elétrico brasileiro já atingiu 144.828 km de linhas de transmissão.  Esse aumento na quantidade de linhas tende a ser amplificado pela tendência à exploração da geração de energia na região Norte, assim sendo necessário a construção de novas linhas para distribuir essa energia para as outras partes do País.

Quanto mais linhas de transmissão e maiores distâncias entre os centros geradores, maiores tendem a ser as perdas. Isso faz com que seja necessária um controle da qualidade dessa transmissão, o que se dá por meio de inspeções. A estrutura já existente apresenta precariedade em alguns aspectos, onde segundo \cite{rangel2009sistema} “no Brasil, há uma quantidade considerável de linhas de transmissão que já ultrapassou os 40 anos de idade. Com o envelhecimento das linhas de transmissão, a manutenção preventiva é um fator de extrema relevância para garantir o perfeito funcionamento dos sistemas.”
A necessidade da constante manutenção e a alta periculosidade que os operadores são expostos faz com novas alternativas e tecnologias sejam aplicadas para a manutenção, o uso de Drones pilotados remotamente, com câmeras e sensores já é uma realidade em alguns países do mundo. O desenvolvimento de um robô próprio para inspeção de linha configura uma dessas novas alternativas, e possibilita uma expansão dos horizontes para as tecnologias aplicadas.
%--------- NEW SECTION ----------------------
\section{Objetivos}
\label{sec:obj}
O objetivo do trabalho é implementar o sistema de movimentação do robô ELIR (\textit{Electrical Line Inspection Robot}). Onde esse sistema é complementar aos outros existentes no robô, onde o conjunto dessas soluções busca fundamentar a implementação de uma Inspeção autônoma da linha.


\subsection{Objetivos Específicos}
\label{ssec:objesp}
Para o desenvolvimento do sistema é necessário realizar o estudo da movimentação, gestão de energia, controle e elaboração de trajetória para sistemas robóticos. A operação na linha faz com que seja necessário a gestão de energia do robô, assim como a integração com os outros subsistemas já desenvolvidos. De forma a garantir a operação na linha, os dispositivos e ferramentas utilizadas devem estar integradas no ROS (\textit{Robot Operating System}), onde é necessário também a integração com outros pacotes já desenvolvidos para o Robô.

%--------- NEW SECTION ----------------------
\section{Justificativa}
\label{sec:justi}


%--------- NEW SECTION ----------------------
\section{Organização do Documento}
\label{section:organizacao}
